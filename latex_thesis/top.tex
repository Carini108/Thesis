In this chapter we elucidate the model used to study the dynamics of a colloidal particle on a corrugated substrate in the presence of a viscoelastic bath.

This chapter is divided into three parts. In the first part, we recall the basics of the Prandtl-Tomlinson model, which has been extensively studied in recent years, see Ref.\cite{vanossi2013}. 
In the second part, we introduce the Langevin equations; this section is based on Ref.\cite{RKubo_1966}. 
The third section formulates an extension of the Prandtl-Tomlinson model that includes a simple implementation of a viscoelastic bath, therefore adding memory to the model, inspired by a recent work \cite{ginot2022} that addressed the simpler two-well problem.
\section{The Prandtl-Tomlinson model} \label{PTmodel}
\noindent The Prandtl-Tomlinson (PT) model is one of the most successful and important models for the description of nanofriction. 
This model is particularly used in friction force microscope (FFM), where friction forces are measured by an atomic force microscopy (AFM) tip that is dragged along a surface.

In the Prandtl-Tomlinson model, the AFM tip is mimicked by a point mass dragged by a spring of elastic constant $K$, which couples the position of the point mass and the position of a FFM support stage driven with a constant velocity $v$. The interaction between the point mass and the substrate over which it is dragged is described by a one-dimensional sinusoidal potential, representing the surface energy corrugation.
This potential is characterized by energy amplitude denoted by $U$ (thus a barrier height equal to $2 U$) and a lattice periodicity represented by $a$.
The corrugated potential and the dragging spring can be combined into a total potential experienced by the point mass 
\begin{equation}
    V(x,t) = U \cos \left(\frac{2\pi}{a} x\right) + \frac{1}{2}K (x-vt)^2 \, .
\end{equation}
The PT model dissipates the energy pumped into the system by the driving stage through a damping viscous force 
\begin{equation}
    F = - \gamma \dot{x}\, ,
\end{equation} 
where $\gamma$ is a damping rate that characterises the energy dissipated effectively into the substrate.
\begin{figure}
    \centering
    \begin{tikzpicture}[>=stealth, scale=0.7]
        \shade[top color=white, bottom color=blue!20!white] (0,{-2+\mincos}) rectangle (6*pi,{-2+\maxcos});
        \draw[domain=0:6*pi, smooth, samples=200, thick, blue!40!white] plot (\x, {cos(deg(\x))});
        \draw (6*pi,1) node[above] {$V(x)$};
        \draw (2*pi,-0.5);
        \draw[<->] (-0.2,-1) -- (-0.2,1) node[above, left] {$2U$};
        \filldraw[ball color=lightblue2] (pi,-1) circle (18pt) node[below] {};
        \node at (pi,-1) {$\gamma$};
        \filldraw[lightblue3] (6,1.5) rectangle (8,2.5);
        \draw[green!20!black, thick] (6,1.5) rectangle (8,2.5);
        \draw[spring] (1.14*pi,-0.55) -- (6,2);
        \node[above=1cm] at (1.5*pi, -0.3) {$K$};
        \draw[->] (8,2) -- (10,2) node[midway, above] {$v$};
        \draw[<->] (3*pi,-2) -- (5*pi,-2) node[midway, below] {$a$};
        \draw[->] (0,-3) -- (6*pi,-3) node[right] {$x$};
        \node[below] at (0,-3) {$x=0$};
        \draw (0,-3) -- ++(0,0.15);
        \draw (0,-3) -- ++(0,-0.15);
\end{tikzpicture}
\caption{Scheme of the Prandtl-Tomlinson model}
\end{figure}

We now introduce the dimensionless parameter $\eta$, defined as
\begin{equation}
    \eta = \frac{4 \pi^2 U}{Ka^2}\, ,
    \label{eta}
\end{equation}
that combines the corrugation amplitude and the characteristic elastic energy of the driving spring.

The Prandtl-Tomlinson model predicts two different patterns of motion depending on the parameter $\eta$
\begin{enumerate}
    \item Smooth sliding regime, which occurs when $\eta < 1$. 
    \\The total potential $V(x)$ shows a single minimum and the sliding of the point mass/tip is smooth over the sinusoidal potential.
    \item Stick-slip regime, when $\eta > 1$. 
    \\In this case, the total potential $V(x)$ exhibits at least two minima, and the sliding becomes intermittent: the point mass stands in one of the minima for a finite time, then rapidly drops into the adjacent minimum.
\end{enumerate}
In the stick-slip regime, it is possible for the particle to exhibit both single-slip and multiple-slip dynamics, meaning it may hop more than one barrier with a single jump, see Ref.\,\cite{PhysRevLett.97.136106}. In this work we are going to analyze  the overdamped regime, characterized by high damping coefficients $\gamma$ (see Ref.\,\cite{Paronuzzi_Ticco_2016}),where inertial
effects become negligible: in this regime, the model does not show multiple-slips.
\\
At finite temperature $T$ due to thermally activated barrier jumps, the overdamped Prandtl-Tomlinson model exhibits a time-averaged friction force $F_\text{k}(v)$ depending on the driving velocity $v$ of the slider, according to the following equation:
\begin{equation}
    F_\text{k} (v) = F_0 - aT^\frac{2}{3} \ln{ \left(b\dfrac{T}{v}\right)}^\frac{2}{3} .
    \label{eq:friction}
\end{equation}
Here $F_0 = F(T=0)$ represents the athermal low-velocity limit of friction, as explained in \cite{vanossi2013}. Equation \eqref{eq:friction} holds for low, but not too low, velocities, whereas for high velocities, the friction force varies linearly with the slider velocity $v$. The static friction force $F_\text{static}$, which represents the force needed to initiate motion between two contacting bodies at rest, is relevant in condition of 'no-sliding' and zero temperature, and it is determined by the derivative of the potential $V(x)$ at its steepest point, which occurs halfway between a minimum and a maximum $x_\text{half} = \frac{3}{4}a$ of the corrugation potential. The static friction force is thus given by the following equation
\begin{equation}
    F_\text{static} = \Bigg\vert\dfrac{\partial }{\partial x} \left(U \cos{\left(\dfrac{2\pi}{a}x\right)}\right)\Bigg\vert_{x=x_\text{half}} = \Bigg\vert \dfrac{2\pi}{a} U \sin{\left(\dfrac{3}{2} \pi\right)}\Bigg\vert = 2 \pi \hspace{0.1cm}Ua^{-1}.
\end{equation}
\section{Brownian motion and Langevin equation} \label{Langevineq}
Brownian motion is a physical phenomenon describing the random motion of a particle suspended in a fluid. This phenomenon is caused by the interaction between the fluid particles and the suspended particle. The fluid particles move randomically due to the effect of temperature, and these random collisions cause random accelerations of the suspended particle.

The Langevin equation is essential in describing the dynamics of particles experiencing stochastic forces, in particular, it is a fundamental tool to understand the Brownian motion of particles.
\\
The standard form of Langevin equation for a particle moving in one dimension is given by
\begin{equation}
    m \ddot{x}(t) = - \gamma \dot{x}(t) + f(x) + \xi(t) \, .
    \label{LangEq}
\end{equation}
Here $m$ represents the mass of the particle, $x(t)$ is the position, $f(x)=-\frac{dV}{dx}$ denotes conservative part of the force acting on the particle, $\gamma$ is the damping coefficient that represents the interaction with the surrounding medium, $\dot{x}$ is the velocity, $\ddot{x}$ is the acceleration and $\xi(t)$ is the stochastic force.

The Gaussian-distributed random force $\xi(t)$ is required to satisfy the fluctuation-dissipation theorem, which can be mathematically expressed as
\begin{equation}
    \langle \xi(t)\xi(t') \rangle = 2 k_B T \gamma \delta(t-t')\, ,
\end{equation}
where $\langle \xi(t)\xi(t') \rangle$ denotes the correlation function of the random force, $k_B$ is the Boltzmann constant, $T$ is the temperature of the system and $\gamma$ is the damping coefficient. The Dirac delta $\delta(t-t')$ indicates that the fluctuations are uncorrelated at different times, indicating that the Langevin thermostat has no memory.

The Langevin equations may be rewritten for the overdamped regime, where the motion of the particle is dominated 
by damping forces and the inertial effects become negligible compared to the dissipative forces.
\begin{equation}
    \gamma \dot{x}(t) = f(x) + \xi(t)\, .
\end{equation}
The overdamped regime is relevant in systems where inertia plays a minor role compared to the dissipative forces, such as the system we are going to consider.
\section{Our model} \label{ourmodel}
The aim of this section is to illustrate the extension of the PT model that we are going to investigate throughout this thesis. The model is an extension of the Prandtl-Tomlinson model when we consider the point mass as a colloidal particle on a corrugated substrate in the presence of a viscoelastic bath.

As precisely described in \cite{Wineman_2009} viscoelastic materials exhibit a combination of viscous, fluid-like, and elastic, solid-like, properties, showing a nontrivial time-dependent behavior when subjected to stress or strain.

A non-Markovian fluid produces a thermostating behavior that deviates from the standard Langevin thermostat in ways that we can describe as follows \cite{Wineman_2009}
\begin{enumerate}
    \item the current state of stress/strain depends not only on the present conditions but also on the past history of the material. Thus also its future behavior is influenced by the sequence of past deformation events.

    \item As a consequence, a non-Markovian thermostat can exhibit a strongly frequency-dependent response, which is more similar to that of an elastic solid when stimulated at high frequency, and more similar to that of a viscous fluid at low frequency.
\end{enumerate}
\begin{figure}[ht]
    \centering
    \begin{tikzpicture}[>=stealth, scale=0.7]
        \shade[top color=white, bottom color=blue!20!white] (0,{-2+\mincos}) rectangle (6*pi,{-2+\maxcos});
        \draw[domain=0:6*pi, smooth, samples=200, thick, blue!40!white] plot (\x, {cos(deg(\x))});
        \draw (6*pi,1) node[above] {$V(x)$};
        \draw (2*pi,-0.5);
        \draw[<->] (-0.2,-1) -- (-0.2,1) node[above, left] {$2U$};
        \filldraw[ball color=lightblue1, opacity=0.3] (1.5*pi,0) circle (32pt) node[below] {};
        \node at (1.5*pi,0) {$\gamma_b$};
        \filldraw[ball color=lightblue2] (3*pi,-1) circle (18pt) node[below] {};
        \node at (3*pi,-1) {$\gamma$};
        \filldraw[lightblue3] (12,1.5) rectangle (14,2.5);
        \draw[green!20!black, thick] (12,1.5) rectangle (14,2.5);
        \draw[spring] (3.14*pi,-0.55) -- (12,2);
        \node[above=1cm] at (3.5*pi, -0.3) {$K$};
        \draw[spring2] (2.8*pi,-1) -- (1.84*pi,-0.3);
        \node[above=1cm] at (2.3*pi, -3.6) {$k_b$};
        \draw[->] (14,2) -- (16,2) node[midway, above] {$v$};
        \draw[<->] (3*pi,-2) -- (5*pi,-2) node[midway, below] {$a$};
        \draw[->] (0,-3) -- (6*pi,-3) node[right] {$x$};
        \node[below] at (0,-3) {$x=0$};
        \draw (0,-3) -- ++(0,0.15);
        \draw (0,-3) -- ++(0,-0.15);
    \end{tikzpicture}
    \caption{Sketch of the non-Markovian Prandtl-Tomlinson model}
    \label{extendedPT}
\end{figure}
Figure \ref{extendedPT} displays a scheme of our model, where the viscoelastic bath is mimicked by the addition to the standard Langevin thermostat of a fictitious particle, characterized by a larger damping coefficient $\gamma_b$, elastically coupled, through a spring constant $k_b$, with the colloidal particle.

In light of these considerations, the generalized Langevin equations can be rewritten as follows
\begin{equation} \label{equations}
    \begin{split}
        &\gamma \dot{x}(t)= -k_b (x - x_b) - \nabla_x V + \xi (t)\\
        &\gamma_b \dot{x}_b(t)= -k_b (x_b - x) + \xi_b (t)
    \end{split}\, .
\end{equation}
\\
The elastic coupling term with the bath particle describes the non-Markovianity of the environment: the colloidal particle position is influenced by its past positions, through the "memory" kept by the $x_b$ fictitious particle.
\\
The equations \eqref{equations} for the potential given by the PT model are 
\begin{equation} \label{eq_def}
    \begin{split}
        &\gamma \dot{x}(t)= -k_b (x - x_b) + \frac{2\pi}{a} U \sin \left(\frac{2\pi}{a} x \right) - K(x-vt) + \xi (t)\\
        &\gamma_b \dot{x}_b(t)= -k_b (x_b - x) + \xi_b (t)
    \end{split}\, .
\end{equation}
As for the standard Langevin thermostat of Section \ref{Langevineq},
\begin{center}
    $\langle \xi_i (t) \rangle = 0 \hspace{0.8cm}
    \langle \xi_i (t) \xi_j (t') \rangle = \delta_{ij} 2 k_B T \gamma_i \delta(t-t') \hspace{0.8cm}i=1,2
$
\end{center}
These equations express that $\xi$ and $\xi_b$ are uncorrelated random forces with zero mean. The fluctuation amplitude of the random forces expressed by the second relation above is such that it guarantees the correct canonical sampling at temperature $T$.
\begin{table}[ht]
\centering 
\begin{tabular}{ll}
    \toprule
    Physical quantity & Units \\
    \midrule
    length    &  $a$   \\
    damping coefficient    &  $\gamma$   \\
    energy & $U$ \\
    force & $U a^{-1}$ \\
    spring constant & $U a^{-2}$\\
    velocity & $U a^{-1}\gamma ^{-1} $\\
    mass & $ U^{-1} a^2 \gamma^2$ \\
    time & $t_0 \equiv U^{-1}a^2\gamma $\\
    \bottomrule
\end{tabular}
\caption{Physical quantities of this work expressed as a combinations of the three natural units of our model: $a$, $\gamma$, $U$.}
\label{tab:physical_quantities}
\end{table}
\\
The model just presented involves several dimensional physical quantities. Given the simplicity of this model, it is convenient to express all physical quantities of this work in terms of natural units: lenghts units of period of the potential corrugation; energies expressed in units of the corrugation amplitude and damping coefficients expressed in units of that characterising the real-particle bath.
\\
Through this work we are going to use both $t_0$ $U^{-1}a^2\gamma$ as the unit of time.
