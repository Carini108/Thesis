Since its introduction in 1928-1929 by L. Prandtl and G.A. Tomlinson, the eponymous model has been the subject of countless theoretical studies in the field of condensed matter physics. The Prandtl-Tomlinson model has thus been acclaimed as one the simplest and most popular model for describing atomic-scale friction.

citazioni di prova: TOP QUARK review del pdg \cite{pdg2021} + quella sperimentale ma va be \cite{Husemann2017}; SIMULATIONS \cite{Alwall2011} and for numbering scheme \cite{pdg2009montecarlo}; ANGULAR OBSERVABLES \cite{Czakon2021} and lastly PROBE FOR ENTANGLEMENT \cite{Severi2022} and \cite{CMS2024} 

Inspired by the article \lq\lq
Barrier Crossing in a Viscoelastic Bath\rq\rq  \hspace{0.1cm}by Ginot \textit{et al}. \cite{ginot2022}, in this work we propose an extension of the Prandtl-Tomlinson (PT) model to include a simple model for a viscoelastic bath, namely an environment characterized by memory effects, what is commonly known as a non-Markovian behavior. These processes are studied in the field of statistical physics, as referenced in \cite{papoulis2002probability}. We describe the dynamics of a colloidal particle in viscoelastic environment by coupling it with a fictitious particle, called \textquotedblleft bath particle\textquotedblright, characterized by its damping coefficient, using an elastic spring.

The first part of this thesis focuses on discussing the standard Prandtl-Tomlinson model and the Langevin equation that describes the Brownian motion of particles. Next, we present a simple non-Markovian extension of the model, with the equations governing its dynamics. The second part of the thesis outlines the method for solving our equations and discusses the choice behind the selection of parameter ranges analyzed throughout the work. The final part of the thesis presents the obtained results and their analyses conducted. 

The initial study focuses on the effective potential experienced by the particle under equilibrium conditions, corresponding to pure Brownian diffusion. The aim of this section is to understand the effect of a non-Markovian environment on the effective potential experienced by the colloidal particle.

The second section focuses on the thermally diffusive undriven model, and in particular on the distributions of waiting times of the particle in potential minima and the dynamics of thermally activated barrier crossing. We evaluate numerically these distributions for pure Brownian diffusion and for non-Markovian Brownian diffusion. A brief discussion on the effect of temperature in this scenario is also provided. Subsequently, the distributions are examined in driven conditions, for both the standard PT model and its extension with a non-Markovian environment. In this latter part, separate analyses are conducted on the distributions of waiting times before a barrier crossing occurs to the right or to the left.

The final study focuses on investigating the velocity-dependence of the friction force, comparing the standard PT model with its non-Markovian extension. Specifically, the high-velocity regime is examined, followed by a preliminary analysis of the more interesting regime of intermediate-to-small velocity.