In this thesis we investigate the overdamped dynamics of a colloidal particle in a viscoelastic bath, namely an environment characterized by memory
effects. From our simple implementation to include a non-Markovian environment in the Prandtl-Tomlinson model, some results and considerations for future developments have emerged, which we summarize here. 

We found that the effect of the viscoelastic bath is an increase of the potential barrier experienced by the particle compared to the standard Brownian case.
This increase of the effective barriers affects the waiting-time distributions. This effect is clearly visible at the level of the distribution resulting from standard Brownian diffusion and non-Markovian Brownian diffusion. In the standard PT model the waiting-time distribution follows an exponential decay. Instead, in the non-Markovian case, we observe a sum of two exponentials, with two radically different characteristic times. At short times, the distribution exhibited a very rapid decay reflecting the fast barrier crossing events caused by the coupling with the viscoelastic bath. In the longer timescale, the distribution shows that the particle can remain in a potential well for many time units before a barrier crossing event occurrs. Our study reveals how a stronger coupling of the particle with the viscoelastic bath leads to an increase in the characteristic time on the long timescale, whereas on the short timescale, the characteristic time is reduced due to the larger restoring force exerted by the bath particle. Furthermore, we observe that the characteristic long timescale decreases with increasing temperature. 

The waiting-time distributions for the standard Prandtl-Tomlinson model and for its non-Markovian extension at finite driving velocity are also quite instructive. Specifically a non-monotonic distribution, with a peak forming around the typical time spent in a potential minimum, determined by the ratio $a/v$ (the washboard frequency) between the distance of two consecutive minima and the dragging velocity.

The main nontrivial features brought to the waiting-time distribution by the non-Markovian model are: (i) the survival of a short-time fast exponential decay related to the fast back-and-forth events promoted by the viscoelastic nature of the thermostat; (ii) a far slower exponential decay of long residence times compared to the regular memory-free model. A separation between the forward (rightward) and backward (leftward) jump events also shows a quite distinct features brought by the memory thermostat.

Finally, we carried out a preliminary investigation of the velocity dependence of the time-averaged friction force in both the standard PT model and its non-Markovian extension. We covered in some detail the trivial high-velocity regime, where both exhibit a linear trend, albeit with a different slope, associated to the extra viscosity brought in by the fake particle modeling viscoelastic effects. We also carried out a preliminary analysis of the low-velocity regime, which proves to be more computationally challenging.

This thesis provides a few important, but preliminary milestones to the investigation of memory effects in the PT model.

We have shown that the simple addition of the viscoelastic bath increases both the effective corrugation, relevant at low velocity, but also the high-velocity viscous friction. For a fair comparison with regular memory-free model an important step will require determining a recipe for the model parameters that would allow a fair comparison of the two thermostats. 

Once this task is achieved, several extension of the present investigation are envisageable. In particular it will be interesting to verify if in the small-velocity regime the regular logarithmic dependence of friction on velocity is retained or modified. However for this task very long simulations will be necessary.